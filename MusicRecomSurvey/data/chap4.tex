\subsection{肖朝军}
小组分工:模型架构搭建;新模型设计,实现与测试;DIN模型的复现与测试。

本次小学期我们小组选择了“社交媒体中的音乐推荐”一题。该项目主要涉及从人们社交媒体中的交互信息中获取相应的用户画像,提取用户兴趣,并为用户推荐相应的音乐。

由于具备神经网络相关的模型设计的经验,在本次项目中我主要负责了数据处理、baseline构建、自主模型设计等工作。为了保证项目选题可以得到圆满的完成,在整个小学期中,我对该项目投入了较大的时间。在前期阅读了大量的推荐系统相关论文之后,我们最终聚焦在推荐系统的冷启动问题上。

在进行文献调研工作时,我发现推荐系统作为目前应用及其广泛的系统,近几年来已经有非常非常多的学者关注到这样一个问题上,近几年光是运用神经网络方法来解决推荐系统问题的文章更是多达几百篇。这也就导致了在短时间内,我们无法对推荐系统的发展现状有一个全面的认识。因此在阅读了助教推荐的几篇文章并和助教沟通后,我们决定着重调研推荐系统冷启动这样一个方向。经过调研,发现研究冷启动问题的论文相对较少,我们也因此决定研究这样一个问题。

在着手实现相关模型时,由于对该领域不够熟悉,我们浪费了很多时间在数据处理上,在负例选取、数据归一化上耽误了很长一段时间,一度觉得自己要做不出来了。后来在助教的帮助下,在中期答辩前,顺利跑完了相关的实验,并验证了我们设计的模型的效果。

在设计模型时,我们主要将自然语言处理领域非常常用的self-attention机制与multihead-attention机制应用到推荐系统中,取得了一个非常不错的效果。因此,这也说明了在进行相关学术研究时,我们不应该局限在某一个狭小的领域,多阅读不同领域的文章可以很好地帮助我们开展相关的工作,并提供灵感。

总而言之,这次小学期项目设计中,遇到了很多的困难,也迸发了很多灵感。我认识到了我许多的不足,在未来的学习过程中,我将注重培养个人的发现问题能力、实验方案设计的能力,努力做出有影响力有意义的学术工作。
感谢助教和老师在本次项目完成中,给我们小组提供的帮助!

\subsection{曹晨阳}
小组分工:Demo架构与经典方法的设计与实现;CML模型的复现;报告撰写。

经过几周的探索和实践,我们最终还是完成了这样一个音乐推荐系统,并取得了不错的成果。

在经过第一周的文献综述后,我们对于数据有一些初步的处理和认识,此时我们对于如何使用这些数据,如何构建我们的模型都充满了疑惑。在第二周,我们决定先从冷启动出发,探究是否有一些解决冷启动问题的思路。我们先后看了一些其他领域中关于冷启动的文章,像是关于Meta-Learning,Metric-Learning等领域,我们发现有些被用于其他领域如CV中的技术可能可以帮助到我们,所以我们逐步参考其他的模型,建立了我们基于Relation-Network的模型。之后就是实验设计等方面的内容了,其中的难点在之前的报告中已经进行了叙述。最后我们的模型在自己的数据集上比现有其他模型取得了更好的效果,说明我们的设计还是有一定的道理的。

之后是一些反思和展望。在快要结题的时候,我们小组的肖朝军同学发现了一篇文章,说当前推荐系统中广泛使用的神经网络方法其实是不必要,甚至有些方法是难以复现的,使用传统的方法也能达到相似的准确度。这也引发了我们小组的思考,我们开始更加注重对于其他baseline模型的复现,以及一些模型设计上的问题。最后的Case Study我们进行了一些工作,但是并不充分,有些可视化也没来的及做,还是有一些遗憾的。在答辩过程中,老师给我们组的意见是更加充分使用当前的数据集,我们认为还是有一些道理的。在最开始的调研当中,我们认为情绪确实能够影响到音乐的选择,但是朋友圈文本量大,图片获取也比较复杂,同时夹杂很多没有意义的文本,所以我们对于这部分的信息利用并不是特别充分,如果我们能够对于有效的朋友圈的情感信息进行分析,相信能够对我们的模型有更大的提升。

最后感谢贾老师和沈导在这几周内的悉心指导。

\subsection{张泽阳}
小组分工:NCF、DeepFM、xDeepFM模型的复现与测试。

在为期五周的小学期中,我们进行了文献综述、切入点探究、模型设计、baseline实现、case study等一系列工作。由于时间紧迫,无法更全面和深入地了解音乐推荐系统,我们选择了“音乐推荐系统中的冷启动问题”作为主要命题。一整套流程下来,对于科研方法有了一定的了解。

在本次项目中,遇到了很多的困难,同时这也伴随着很大的收获:
前两周文献综述的时候,看了大量文章,但始终类似于一个BFS的过程,看在眼里,觉得挺有道理,但是由于之前没有接触过这个领域和模型实现,也难以断定其优点与可改进之处。总觉得一个模型是理论上要体现出优越性,才能对实验有所期待,但实际上这会导致进度遥遥无期。快开题的时候,还是朝军同学敏锐地发现了Relation network 在本命题上的可改进性,一针见血地提出了其关于冷启动问题对症下药的优越性,让我们如期开题。

在实现NCF的过程中,一开始诧异于输入的数据只有id或者交互矩阵的行列,最后和同学交流才发现其本质上还是CF方法,只不过神经网络不同于传统方法如MF、FM等,能学到更高阶的关系。看NCF论文的时候,看到过他们采用了负采样的方法,没多在意。后来和助教沟通才发现,我们的模型AUC很低,是由于我们对于数据本身的理解出现了偏差,在采用负采样之后,终于在报告前回归正常水平。当AUC比较低的时候,本能地想是不是参数不太对,浪费了许多时间在调参上,而实际上数据利用、模型的训练技巧上更值得考究。我感到,模型实现和积极交流能使得研究者对于论文思想的理解更加深刻。

在几次报告中,老师和助教给我们提出了一些建议。我感到,模型具体的参数、指标可能并不是特别重要,而重要的是模型背后的故事——要改进什么?怎么改进的?为什么改进了?我想,这也分别是文献综述、模型设计、case study所要面临的问题。通过团队的合作努力,以及老师助教们的帮助与建议,我觉得我们在这些方面都做了很多,也学到了很多。

总而言之,本次小学期使我受益匪浅,同时也让我意识到之前自身研读论文不深不广、模型实现和方案设计都经验不足、发现问题的能力也有所欠缺。希望以后能在自己追逐的方向上做得更好。
十分感谢一起合作的队友、辛勤付出的老师和助教们!
