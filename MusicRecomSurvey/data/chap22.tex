\subsection{数据集处理}
\subsubsection{数据筛选}
经过对数据的初步查看,我们发现我们的数据集中虽然规模较大,但是其中的噪声比较大,所以我们对于数据集进行了一步初筛。首先,我们筛除了交互数据数量少于10的用户以及音乐。来保证训练集中的样本信息足够充分。同时我们简单筛除了一部分微信中微商的信息,这部分用户及其文本信息我们认为并不能表现出太大的价值。经过这样的一轮筛选,我们得到的数据集有181137个用户,38306首音乐,15258008项交互数据。我们的数据集和其他相关的数据集的比对如表2所示:

\begin{table}[htbp]
\centering
\caption{不同数据集的情况}
\begin{tabular}{|l|l|l|l|}
\hline

Dataset & Users & Items & Samples \\ \hline
Amazon(Electro) & 192,403 & 63,001 & 1,689,188 \\ \hline
MovieLens\cite{Harper:2015:MDH:2866565.2827872} & 138,493 & 27,278 & 20,000,263 \\ \hline
Ours & 181,137 & 38,306 & 15,258,088 \\ \hline

\end{tabular}
\end{table}

\subsubsection{音乐行为处理}
我们将分享和收藏的行为计10分;收听按照时长和总时长的比例积分,完整收听计为5分;摇一摇计为5分。综合计算上面的所有得分,对于高于6分的交互记为正例。

\subsubsection{特征选取}
经过一些测试,我们选定了如下的用户及音乐特征,这些特征信息均能在我们的数据集中得到:
\begin{itemize}
    \item 用户:性别、年龄、省份、文章阅读、朋友圈文本
    \item 音乐:歌手、流派、歌词、音频
\end{itemize}


\subsubsection{数据集构建}
接下来,我们要对数据集进行分类,划分出正负例,以及训练集和测试集。在上面的形式化描述中,我们将问题定义为一个二分类的问题,所以这里我们也采取二分类的方式进行选取。根据上面对于音乐行为的处理评分,我们将高于6分得分的音乐记为正例,这表明所有分享和点赞的音乐均为正例,完整收听并不会被记为正例,但是如果还有摇一摇等行为,则表明用户有一定的兴趣去主动查找对应的音乐,此时则被记为正例。

接下来需要选取负例。我们选择使用负采样的方法进行负例选取:随机在负例样本中选取和正例数量相同的负例,来保证样本分布尽量均衡。但是,什么样的样本算是负例,就是我们需要考虑的问题了。首先我们将低于6分的样本记为负例,在其中进行随机采样。但是经过一些实验和预训练,我们发现这样训练得到的模型的效果并不好。之后,我们经过讨论认为不能将分数小于6分的音乐视作负例。因为有评分的音乐表明用户和这个音乐有一定的交互行为,虽然没有表现出明显的兴趣,但是这种交互行为表现出了一个正向的倾向。所以我们认为将这部分设置为负例是不合适的。经过调整过后,我们将用户未进行评分的音乐作为负例集合,在其中进行随机采样。虽然这部分的音乐中可以看作是含有用户感兴趣的内容的,但是我们认为相比于这些感兴趣的内容,在这部分音乐中用户不感兴趣的内容数量是更多的。所以这样的采样有很大的概率采样得到的就是不喜欢的内容了。

完成正负例的判定和选取之后,我们歌曲按照4:1的方式对于训练集和测试集进行划分,保证测试音乐只在测试集中出现。此时在训练过程中就不会出现这些测试集的音乐,因此在测试时,这些音乐相当于全新的音乐,成功模拟了冷启动的情况。此时,有30645首歌曲在训练集中,7661首歌曲在测试集中。