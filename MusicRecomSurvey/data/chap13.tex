\subsection{其他工作与计划}
在确定了模型的基本形式的同时,我们也开始对于一部分的BaseLine进行实验,并进行了一部分的准备工作。

\subsubsection{处理数据}
我们首先对于手中的数据集进行了一定的处理:对于用户特征,我们选择使用23维的公众号阅读量来进行表征,并进行了相应的处理;对于音乐特征,我们选取了歌词的LDA,音频信息,歌手,歌曲类型等数据;对于交互信息,我们使用评分来描述用户对歌曲的行为,例如根据用户听歌的时长,是否收藏,是否转发等指标,将其转化为一个0~15的评分,来表示用户对于歌曲的喜好程度;此外,我们对于数据集进行了一定的划分,得到了训练集和测试集。

\subsubsection{BaseLine}
我们选取了三个BaseLine对于实验结果进行比较,分别为:
\begin{itemize}
    \item LRMM\cite{LRMM}
    \item DeepFM\cite{guo2017deepfm}
    \item CML\cite{CML}
\end{itemize}

在之后的工作中,我们希望能够实现三个关于Cold Start的BaseLine,来对于我们模型中处理冷启动的性能进行比较和评价:
\begin{itemize}
    \item Low-Rank Linear Cold-Start Recommendation from Social Data (2017)\cite{sedhain2017low}
    \item Collaborative filtering and deep learning based recommendation system for cold start items (2017)\cite{wei2017collaborative}
    \item From Zero-Shot Learning to Cold-Start Recommendation (2019)\cite{li2019zero}
\end{itemize}

此外,我们在之后也希望能够在其他数据集上对于我们的模型进行验证,如:
\begin{itemize}
    \item LastFM\cite{Zafarani+Liu:2009}
    \item MovieLens\cite{Harper:2015:MDH:2866565.2827872}
\end{itemize}

\subsubsection{DEMO设计}
由于我们当前的关注点更多的在于模型的设计和实现,所以我们希望能够在实现模型的同时,设计一些简单的Demo来展示我们的数据集,例如展示稀疏数据下的推荐效果比对,以及基于音乐相似度的推荐理由。

\subsubsection{Case Study}
由于我们在实验中使用了用户和音乐的高阶特征,这就使得我们在模型的可解释性上有一定的欠缺,所以我们希望能够通过一些Case Study来分析我们的模型的实际情况,并通过分析典型的用户画像,对应其推荐的结果,评价我们的模型。