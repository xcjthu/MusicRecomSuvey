\subsection{文献综述总结}
基于文献的阅读和总结,我们初步了解了推荐系统的实现方法和发展历程,包括早期的传统方法和之后的神经网络方法。在调研过程中,我们发现,神经网络方法确实能够比传统方法有更高的性能,所以我们下一步的思路更倾向于进一步发掘神经网络方法在社交媒体用户的音乐推荐上的应用。但是,这也带来了一些潜在的问题,比如可解释性等,这些问题也是我们在下一步研究中重点关注的部分。经过讨论,我们对于研究方向的初步展望有如下几个方向:
\begin{itemize}
    \item 可解释的推荐系统:为了解决神经网络方法在可解释性上的缺陷,可以对于如何在可解释性上进行进一步的研究,从而提高模型的性能。
    \item 推荐系统的推理能力:由于我们拿到了一系列用户在听歌这一行为之外的数据,如朋友圈的发文等,所以我们希望能够从这些信息中,推理出用户的更多信息,来进行精度更高的推荐。
    \item 跨领域推荐:不同的推荐系统之间,从某种程度上是可以有一定的交互的。因为两者都在一定程度上依赖于用户本身的特征,所以一种可能的方向就是根据其他推荐系统的结果,来进行音乐推荐。
\end{itemize}
更加详细的研究计划将在后面一周的讨论中逐步进行完善和初步构建。