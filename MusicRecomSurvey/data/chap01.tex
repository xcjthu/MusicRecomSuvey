\subsection{背景}
% 介绍推荐发展的历史,以及为什么要有社交媒体的推荐
推荐系统是信息过滤系统的一个子集,核心的功能是根据用户的个人信息和历史行为,对于用户对于一些内容的潜在评价或者评分进行估计,以及将用户潜在的感兴趣的内容放置在对于用户曝光率较高的地方。

推荐系统的发展背景是互联网的发展以及信息量的增长,这就导致了用户在搜索一些内容时所需要的时间越来越多,所以需要较为有效的检索方法。当用户较为准确地知道自己需要看的内容,则可以通过搜索引擎渠道很容易地得到自己所需要的,但是对于一些探索性的内容,如看新闻、听音乐等,用户可能也是在对于相关内容进行探索,选择自己感兴趣的内容去看,所以此时搜索引擎就无法准确满足用户需求了。所以此时如何为用户展现可能感兴趣的优质内容,就是推荐系统的主要任务了。

推荐系统已经运用到了很多应用及服务当中,如电子商务(淘宝、亚马逊等),社交平台(微博、Twitter等)以及内容平台(音乐平台、新闻平台等)。

对于音乐推荐系统,其主要的目的就是根据用户的个人信息以及听的歌曲的特色,为用户推荐可能感兴趣的歌曲或者歌单。与传统的推荐任务相比,其有一个显著的特征就是受到用户个人状态的影响较大,同时可能会有一些社交属性。所以在传统的推荐系统的基础上,考虑将用户在社交平台上的相关信息纳入到考虑范围,能够更加准确地描述用户当前的需求和喜好,更加准确地为用户进行音乐推荐。所以我们的研究主题就是将用户在社交媒体上的相关行为应用在用户音乐推荐上。

