\subsection{Case Study}
通过我们的模型,我们对于不同的歌手之间的相关关系进行了一定的调研,选出了每个歌手最相似的歌手,结果如下:

周杰伦:周杰伦, 张国荣, 赵雷, 任贤齐, 黑龙, 李克勤, 庄心妍, 望海高歌, 张碧晨, 风语

张国荣:张国荣, 周杰伦, 许嵩, 赵雷, 黑龙, 张碧晨, 李健, 凤凰传奇, 任贤齐, 庄心妍

许嵩:许嵩, 凤凰传奇, 张国荣, 张碧晨, 周杰伦, 黑龙, 李健, 赵雷, 任贤齐, 庄心妍

汪峰:汪峰, 郑秀文, 邓紫棋, 崔子格, 网络歌手, 卫兰, 张津涤, 门丽, 蔡健雅, 冷漠

邓紫棋:邓紫棋, 汪峰, 郑秀文, 卫兰, 崔子格, 网络歌手, 蔡健雅, 郑源, 张津涤, 门丽

五月天:五月天, 杨千嬅, 杨坤, 祁隆, 张玮伽, 张杰, 谭咏麟, 张宇, 秋裤大叔, 水木年华

杨坤:杨坤, 五月天, 周传雄, 水木年华, 张杰, 望海高歌, 汪苏泷, 谭咏麟, 杨千嬅, 李克勤

可见,对于歌手之间的相关关系,我们并不能找出特别明显的关联,但是仍然可以部分描述歌手与歌手之间的风格相似程度。比如许嵩和周杰伦的距离较近,而许嵩在早期以翻唱周杰伦起家,风格势必有一定的相似性。同时,汪峰和邓紫棋的曲风也有一定的相似性,杨坤和五月天的音乐风格也是相似的。这表明我们的模型能够从某种层面上对于歌手,歌曲特征进行抽象,但是这种抽象的可解释性较差,很难找到特别明显的特征信息。