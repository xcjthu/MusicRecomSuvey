%%%%%%%%%%%%%%%%%%%%%%%%%%%%%%%%%%%%%%%%%%%%%%%%%%%%%%%%%%%%%%%%%%%%%%%%%%%%%%%%%%%%%%%%%%%%%%
% --一些基本设置,各种包的扩展--
%%%%%%%%%%%%%%%%%%%%%%%%%%%%%%%%%%%%%%%%%%%%%%%%%%%%%%%%%%%%%%%%%%%%%%%%%%%%%%%%%%%%%%%%%%%%%%
% A4纸张大小
\documentclass[UTF8,a4paper,10pt]{ctexart}
% 页边距
\usepackage[left=2.50cm, right=2.50cm, top=2.50cm, bottom=2.50cm]{geometry}
% 设置章标题居左
\CTEXsetup[format={\Large\bfseries}]{section} 
%%%%%%%%%%%%%%%%%%%%%%%%%%%%%%%%%%%%%%%%%%%%%%%%%%%%%%%%%%%%%%%%%%%%%%%%%%%%%%%%%%%%%%%%%%%%%%
% -- 专门设置页眉页脚 --
%%%%%%%%%%%%%%%%%%%%%%%%%%%%%%%%%%%%%%%%%%%%%%%%%%%%%%%%%%%%%%%%%%%%%%%%%%%%%%%%%%%%%%%%%%%%%%
\usepackage{fancyhdr}
\pagestyle{fancy}
% bfseries
% 页眉
\lhead{}
\chead{社交媒体用户的音乐推荐}
\rhead{}
% 页脚
\lfoot{}
\cfoot{第 \thepage 页}
\rfoot{}
% 修改格式
\renewcommand{\headrulewidth}{0.4pt}
\renewcommand{\footrulewidth}{0.4pt}
%%%%%%%%%%%%%%%%%%%%%%%%%%%%%%%%%%%%%%%%%%%%%%%%%%%%%%%%%%%%%%%%%%%%%%%%%%%%%%%%%%%%%%%%%%%%%%
% -- 设置字体 --compile using Xelatex
%%%%%%%%%%%%%%%%%%%%%%%%%%%%%%%%%%%%%%%%%%%%%%%%%%%%%%%%%%%%%%%%%%%%%%%%%%%%%%%%%%%%%%%%%%%%%%
% -- 中文字体 --
%\setmainfont{Microsoft YaHei}  % 微软雅黑
%\setmainfont{YouYuan}  % 幼圆    
%\setmainfont{NSimSun}  % 新宋体
%\setmainfont{KaiTi}    % 楷体
%\setmainfont{SimSun}   % 宋体
%\setmainfont{SimHei}   % 黑体
% -- 英文字体 --
\usepackage{times}
%\usepackage{mathpazo}
%\usepackage{fourier}
%\usepackage{charter}
%\usepackage{helvet} 
%%%%%%%%%%%%%%%%%%%%%%%%%%%%%%%%%%%%%%%%%%%%%%%%%%%%%%%%%%%%%%%%%%%%%%%%%%%%%%%%%%%%%%%%%%%%%%
% --设置水印--
%%%%%%%%%%%%%%%%%%%%%%%%%%%%%%%%%%%%%%%%%%%%%%%%%%%%%%%%%%%%%%%%%%%%%%%%%%%%%%%%%%%%%%%%%%%%%%
% 所有页加水印
%\usepackage{draftwatermark}
% 只有第一页加水印         
%\usepackage[firstpage]{draftwatermark}
% 设置水印内容 
%\SetWatermarkText{Water-Mark}
% 设置水印logo           
%\SetWatermarkText{\includegraphics{fig/ZJDX-WaterMark.eps}}
% 设置水印透明度 0-1         
%\SetWatermarkLightness{0.9}
% 设置水印大小 0-1             
%\SetWatermarkScale{1}
%%%%%%%%%%%%%%%%%%%%%%%%%%%%%%%%%%%%%%%%%%%%%%%%%%%%%%%%%%%%%%%%%%%%%%%%%%%%%%%%%%%%%%%%%%%%%%
% --用于标注代码的样式--
%%%%%%%%%%%%%%%%%%%%%%%%%%%%%%%%%%%%%%%%%%%%%%%%%%%%%%%%%%%%%%%%%%%%%%%%%%%%%%%%%%%%%%%%%%%%%%                     
\usepackage{listings} 
\lstset{						 % 下面是一些设置
  language=python,               % 代码语言
  basicstyle=\footnotesize,      % 代码的尺寸大小
  numbers=left,                  % 将行号写到左边
  numberstyle=\tiny\color{gray}, % 行号的风格
  stepnumber=2,                  % 行与行的间隔标号,如果为1,则每一行都会被标记行号
  numbersep=5pt,                 % 行号与代码之间的间距
  backgroundcolor=\color{white}, % 如果选择背景色,你必须加 \usepackage{color},color可以是其他色
  showspaces=false,              % show spaces adding particular underscores
  showstringspaces=false,        % underline spaces within strings
  showtabs=false,                % show tabs within strings adding particular underscores
  frame=single,                  % adds a frame around the code
  rulecolor=\color{black},       % if not set, the frame-color may be changed on line-breaks 									within not-black text (e.g. commens (green here))
  tabsize=2,                     % sets default tabsize to 2 spaces
  captionpos=b,                  % sets the caption-position to bottom
  breaklines=true,               % sets automatic line breaking
  breakatwhitespace=false,       % sets if automatic breaks should only happen at whitespace
  title=\lstname,                % show the filename of files included with \lstinputlisting;
                                 % also try caption instead of title
  keywordstyle=\color{blue},     % keyword style
  commentstyle=\color{dkgreen},  % comment style
  stringstyle=\color{mauve},     % string literal style
  escapeinside={\%*}{*)},        % if you want to add LaTeX within your code
  morekeywords={*,...}           % if you want to add more keywords to the set
}
%%%%%%%%%%%%%%%%%%%%%%%%%%%%%%%%%%%%%%%%%%%%%%%%%%%%%%%%%%%%%%%%%%%%%%%%%%%%%%%%%%%%%%%%%%%%%%
%%%%%%%%%%%%%%%%%%%%%%%%%%%%%%%%%%%%%%%%%%%%%%%%%%%%%%%%%%%%%%%%%%%%%%%%%%%%%%%%%%%%%%%%%%%%%%
% --其他可用的建设性模板(仅供复制修改)--
%%%%%%%%%%%%%%%%%%%%%%%%%%%%%%%%%%%%%%%%%%%%%%%%%%%%%%%%%%%%%%%%%%%%%%%%%%%%%%%%%%%%%%%%%%%%%%                    
%%%%%%%%%%%%%%%%%%%%%%%%%%%%%%%%%%%%%%%%%%%%%%%%%%%%%%%%%%%%%%%%%%%%%%%%%%%%%%%%%%%%%%%%%%%%%%
% --用于创建表格--
%\begin{table}[htbp]
%		\caption{Title of table} \label{tab:table}
%		\centering
%		\addtolength{\tabcolsep}{-0mm} % 控制列间距
%		\begin{tabular}{ccccc}
%			\toprule[0.75pt]	% package booktabs
%			\multicolumn{4}{c}{table head} \\
%			\midrule[0.5pt]	% package booktabs
%			\multirow{4}{*}{text} & 1 & 2 & 3 & 4 \\  % package multirow
%			& 5 & 6 & 7 & 8 \\
%			\cmidrule[0.5pt]{2-4}	% package booktabs
%			& 9 & 10 & 11 & 12 \\
%			& 13 & 14 & 15 & 16 \\
%			\bottomrule[0.75pt]	% package booktabs
%		\end{tabular}
%\end{table}
%%%%%%%%%%%%%%%%%%%%%%%%%%%%%%%%%%%%%%%%%%%%%%%%%%%%%%%%%%%%%%%%%%%%%%%%%%%%%%%%%%%%%%%%%%%%%%
% --写个算法伪代码流程图--
%\begin{algorithm}
%		\caption{Title of the Algorithm}
%		\label{algo:ref}
%		\begin{algorithmic}[1]
%			\REQUIRE some words.  % this command shows "Input"
%			\ENSURE ~\\           % this command shows "Initialized"
%			some text goes here ... \\
%			\WHILE {\emph{not converged}}
%			\STATE ... \\  % line number at left side
%			\ENDWHILE
%			\RETURN this is the lat part.  % this command shows "Output"
%		\end{algorithmic}
%\end{algorithm}
%%%%%%%%%%%%%%%%%%%%%%%%%%%%%%%%%%%%%%%%%%%%%%%%%%%%%%%%%%%%%%%%%%%%%%%%%%%%%%%%%%%%%%%%%%%%%%
%%%%%%%%%%%%%%%%%%%%%%%%%%%%%%%%%%%%%%%%%%%%%%%%%%%%%%%%%%%%%%%%%%%%%%%%%%%%%%%%%%%%%%%%%%%%%%

\usepackage{color}
\usepackage{cite}

\definecolor{dkgreen}{rgb}{0,0.6,0}
\definecolor{gray}{rgb}{0.5,0.5,0.5}
\definecolor{mauve}{rgb}{0.58,0,0.82}


 
\usepackage{amsmath, amsfonts, amssymb} % math equations, symbols
\usepackage[english]{babel}
\usepackage{color}      % color content
\usepackage{graphicx}   % import figures
\usepackage{url}        % hyperlinks
\usepackage{bm}         % bold type for equations

\usepackage{multirow}
\usepackage{booktabs}
\usepackage{epstopdf}
\usepackage{epsfig}
\usepackage{algorithm}
\usepackage{algorithmic}


% use Input in the format of Algorithm
\renewcommand{\algorithmicrequire}{ \textbf{Input:}}
% use Initialize in the format of Algorithm       
\renewcommand{\algorithmicensure}{ \textbf{Initialize:}} 
% use Output in the format of Algorithm  
\renewcommand{\algorithmicreturn}{ \textbf{Output:}}       
 
 
 
 
  
% 超链接
%\usepackage{hyperref} %bookmarks
%\hypersetup{colorlinks, bookmarks, unicode} %unicode
% 
% 
% 


%%%%%%%%%%%%%%%%%%%%%%%%%%%%%%%%%%%%%%%%%%%%%%%%%%%%%%%%%%%%%%%%%%%%%%%%%%%%%%%%%%%%%%%%%%%%%%
% --正文如下--
%%%%%%%%%%%%%%%%%%%%%%%%%%%%%%%%%%%%%%%%%%%%%%%%%%%%%%%%%%%%%%%%%%%%%%%%%%%%%%%%%%%%%%%%%%%%%%
%% 一些前期工作
% 文章题目
\title{\textbf{社交媒体用户的音乐推荐文献综述}}
% 作者
\author{ 计63 肖朝军 2016011302 \\ 计63 曹晨阳 2015010603 \\ 计63 张泽阳 2016} %\thanks{学号:2016310200112} }
% 日期(还没想到怎么删)
\date{\today}
%%%%%%%%%%%%%%%%%%%%%%%%%%%%%%%%%%%%%%%%%%%%%%%%%%%%%%%%%%%%%%%%%%%%%%%%%%%%%%%%%%%%%%%%%%%%%%
\begin{document}
    \maketitle %输出标题
        
 
%\newpage%另起一页
% \begin{abstract}
% 文献综述就不要摘要了

%\centering
% \textbf{关键字:}关键词1 关键词2
% \end{abstract}
\thispagestyle{empty}



%标题栏

\section{背景}
% 介绍推荐发展的历史,以及为什么要有社交媒体的推荐


\subsection{音乐推荐系统分析}
虽然音乐推荐系统是推荐系统的一个小部分,但是由于歌曲和收听用户自身的一些特性,音乐推荐系统和其他推荐系统还是有一些不同点的\cite{MusicComSurvey},比如与电影,书籍,新闻等推荐系统。在此部分中,我们将对于这些音乐推荐系统特性,面临的挑战与发展方向进行一定的分析。

\subsubsection{音乐推荐系统的特性}
首先,分析一下音乐及其推荐系统不同于其他推荐任务的特性\cite{MusicComFeature};
\paragraph{时长}
我们对于一个物品或一项内容的感觉与我们接触它的时长时有一定关系的,从时长上来看,通常的电影持续约90分钟或者更长,而对于书籍我们通常要花费若干小时,若干天甚至一周以上的时间来进行阅读。而对于音乐,大部分的流行音乐的时长都在3分钟到5分钟的时间内,而可能存在一些古典音乐,其时长会有更长,但是相比于电影或者是书籍,仍然有一些差距\cite{MusicComSurvey}。因此,较小的持续时长也就导致了音乐表现出更多的“可丢弃”的性质。

\paragraph{数量}
推荐系统也受到其可供推荐的内容多少的限制。对于电影推荐系统的话,根据Netflix在2016年的数据,其电影和电视剧的总数约为5500;而对于书籍推荐系统,亚马逊在2017年的书籍数量为4000万左右;对于音乐推荐,根据Spotify在2017年的数据,报告了约3000万首歌曲。\cite{MusicComSurvey}可见相较于电影推荐系统,歌曲的推荐的集合要大的多,这就要求音乐推荐系统必须拥有更加高效的方法来对于如此庞大的数据进行分析处理。

\paragraph{连续}
用户在听歌时,一次可能会听很多首歌,或者说用户的听歌行为很少是面向于单曲的,而更像是面向于歌单、专辑或者歌手的。所以这种连续的收听与电影或者书籍的推荐系统又有了一些不同。

\paragraph{重复}
对于电影来讲,大部分电影我们在一段时间内可能不会去看很多遍,除非我们对于这个电影有极大的偏爱。但是此时就算推荐系统不给我们推荐我们也会去看,所以在这里不存在重复推荐的问题。但是对于歌曲推荐系统而言,歌曲推荐是存在重复推荐的,那么两次推荐之间的时间间隔应该怎样选取,怎样的音乐适合于进行重复推荐等问题也就相应产生了。

\paragraph{目的}
音乐对于用户来讲,其实是有多种功能的,例如我们可能会在打牌时选择播放“斗地主”的背景音乐来活跃气氛,或者为了哄家里的小孩子放一些儿童向的歌曲。这些播放行为通常带有更多的是我们的目的性,而非我们个人对于这类歌曲有更大的偏爱。所以能否正确辨别这些情况,能够很大地改善用户在使用推荐系统时的体验。此外,有研究人员对于听音乐的目的在社会学的角度上也提供了一些研究\cite{MusicUses}。

\paragraph{心境}
音乐与人内心的情感是密不可分的,相同的人在不同的心情下可能会选择不同的歌曲来听\cite{EmotionInflunce}。所以音乐推荐系统会更加依赖于对于用户情感的分析。同时,音乐对于人的心情也有相互的作用,有些时候我们可能需要音乐来为我们“助兴”,而有些时候我们可能需要音乐来帮助我们放松心情,减轻压力等。这种基于用户心境的推荐是音乐推荐系统中比较关键且富有挑战性的内容。

\paragraph{场景}
与心情类似,场景也是决定我们决定听歌内容的一大影响因素\cite{SituationalInflunce}。如果能够根据用户的地理位置,将这部分信息纳入到音乐推荐的评估当中,也能够提高用户的使用体验\cite{LocationalInflunce}。

\subsection{音乐推荐系统的挑战}
在音乐推荐的过程当中,存在一些独特的,且影响较大的问题\cite{Challange}:

\paragraph{冷启动}
冷启动指的是在没有足够多的用户数据的时候如何对于新用户正常地进行音乐推荐,或者对于新上架的歌曲,没有更多的评价信息,如何将其正确地推荐给用户。冷启动的问题在很多的推荐系统当中是一个普遍存在的问题\cite{ColdStart},但是对于音乐推荐系统来讲,其短时间,多数量的性质导致了这种冷启动的问题很严重。这种冷启动还可能带来的一个问题就是数据的稀疏性,对于一个用户来讲,其未评分的歌曲数远大于其进行过评分的歌曲数。当这种稀疏性过大时,可能会导致推荐系统不准确\cite{Challange}。

\paragraph{自动播放列表}
在上面的特性中可以看出,用户在听歌时经常以一个歌单为基本的形式。所以如何自动生成、维护和推荐歌单就是推荐系统需要考虑的又一大问题。同时,在歌单内部歌曲之间是否具有一定的共同特征,以及歌曲之间的相互衔接对于歌单的质量也会造成一些影响\cite{Context}。在歌单的基础上,歌单与歌单之间的关联与推荐,歌单的冷启动问题,也有一些值得去处理的问题。

\paragraph{评估标准}
在机器学习以及其他相关领域中,已经发展出一系列用于评估性能的参数,在音乐推荐系统中也有较大的效用。这部分内容将在后面进行更加详细的介绍。

\subsubsection{发展前景}
\paragraph{基于用户心理的音乐推荐}
在上面的特性中可以看出,用户的心情对于其音乐的选择是起到很大的影响的\cite{EmotionInflunce}。在传统的方法中,我们用到的更加普遍的方法是根据用户的历史行为,以及一些个人的背景资料,来完善推荐系统对于人的个性化需求的处理。所以对于音乐推荐系统,其一大改进方向就是通过一些用户数据,推测用户当前的心理状态,在此基础上结合其他特征进行音乐的推荐。

\paragraph{基于场景的音乐推荐}
一般的推荐系统的重点均为根据用户以及音乐的相关内容,来求出推荐内容。但是在真实的生活场景中,一些在用户和音乐之外的因素同样能够对于音乐的推荐造成比较大的影响,就如时间\cite{TimeInflunce},地点\cite{Location},季节\cite{Season}等。所以基于场景的音乐推荐能够综合考虑用户实在什么样的场合下播放的音乐,从而能够推荐更加符合当前场景的音乐。

\paragraph{基于文化背景的音乐推荐}
个人的兴趣爱好和选择偏向已经在推荐系统中被高效地利用且有着不错的表现,但是更高层次地,用户所在的文化氛围从一定程度上决定了用户一部分的选择偏向\cite{Culture},所以如果能够现根据用户的文化背景进行一个粒度较粗的分类,再在此基础上进行细分,可以提升系统的准确程度。
\section{基于神经网络的推荐算法}
\section{推荐系统评价指标}


\bibliography{ref}
%\section*{参考文献}

\end{document}
%结束
